\documentclass[12pt]{article}
\usepackage{amsmath, amssymb, amsfonts, amsthm}
\usepackage{graphicx, tabularx, geometry, xcolor, minted}
\usepackage{caption, float}
\usepackage{setspace, lipsum}

\usepackage{tikz, pgfplots}
\pgfplotsset{compat=1.18}

\geometry{top=1 in, bottom=1 in, left=1 in, right=1 in}

% \onehalfspacing

\theoremstyle{definition}
\newtheorem{theorem}{Theorem}
\newtheorem{definition}{Definition}
\newtheorem{lemma}{Lemma}
\newtheorem{proposition}{Proposition}
\newtheorem{remark}{Remark}
\newtheorem{corollary}{Corollary}
\newtheorem{problem}{Problem}
\newtheorem{example}{Example}

% \setlength\parindent{0 pt}

\begin{document}

\begin{titlepage}
    \begin{center}
        \vspace*{2 cm}

        \begin{spacing}{2}
            {\Large \textbf{Polynomial Root-Finding Methods: \\ Theory, Weaknesses, and Wilkinson's Polynomial}}
        \end{spacing}

        \vspace*{1 cm}

        {\large Joel Penney} \par

        \vspace*{0.25 cm}
        
        \texttt{jscottp@mun.ca}

        \vfill

        \begin{abstract}
            This paper studies six algorithms used to solve root-finding problems involving single-variable polynomials. Specifically, we consider theoretical foundations such as the intermediate value theorem as well as the details of iterative rules to explain the differences between several bracketed and open methods. The six algorithms are also compared under challenging circumstances such as flat regions and Wilkinson's polynomial. We find that no algorithm performs consistently better than all of the others, although TOMS748 and Halley's method are often able to locate roots quickly. As we demonstrate, the structure of the root-finding problem influences the chosen approach, relevant theory considered, and the methods applied to efficiently find roots.
        \end{abstract}

        \vfill

        \begin{spacing}{1.5}
            Memorial University of Newfoundland \par
            Department of Mathematics and Statistics \par
            MATH 3030: Mathematical Inquiry II
        \end{spacing}

        \vspace*{1.25 cm}

        January 29, 2026

        \vspace*{0.25 cm}
    \end{center}
\end{titlepage}

% \tableofcontents
% \newpage

\section{Introduction}
\noindent
Suppose you are a physicist, studying the trajectory of a projectile as it flies through the air. Alternatively, picture yourself as a biological researcher trying to model the spread of disease in a population. Perhaps, in another case, you want to analyze costs and predict economic patterns to ensure your financial decisions are well-informed. In all of these cases, polynomials are used to represent a real situation or phenomena. ...

\section{Methods}

\subsection{Preliminaries}
\noindent
Suppose we know a function $f$ and want to determine the value or values of $x$ such that \[f(x)=0.\]

We may first try analytical methods to determine $x$, which is often convenient if $f$ is sufficiently basic, such as a polynomial of degree 1 or 2. Frequently, however, our function $f$ is more complicated or sophisticated, and analytical methods are impractical or unavailable. Instead, numerical methods are often imposed on \textit{root-finding problems} such as the one described. Root-finding algorithms (or root-finding methods) are used to approximate the roots or zeros of a function, with the requirement that the function be continuous. Additionally, solving the equation \[g(x)=h(x)\] is equivalent to finding the roots of $f(x)=g(x)-h(x)$, which shows how widespread root-finding problems are and how important root-finding methods can be. 

\begin{definition}
    A \textit{polynomial of one variable}, denoted $p(x)$, is an expression that can be written in the form \[a_nx^x + a_{n-1}x^{x-1} + \cdots +a_1x + a_0\] where $a_n, ..., a_0$ are coefficients, $a_n$ is nonzero, $x$ is a variable, and $n$ is the degree \cite{kalantari2008polynomial}. If the coefficients are real and the variable is restricted to the real numbers, then the expression is called a \textit{real polynomial}. The equation $p(x)=0$ is called a \textit{polynomial equation}.
\end{definition}

Each polynomial of one variable (or just polynomial for short) is continuous everywhere.

\begin{theorem}[Fundamental Theorem of Algebra]
    Every nonzero polynomial of one variable with complex coefficients has exactly $n$ complex roots when multiplicity is counted.
\end{theorem}

Many proofs of this famous theorem exist. They can be found in  \cite{steed2015proofs}, for example. Although we are primarily interested in \textit{real} roots, the theorem indicates exactly how many (complex) roots a polynomial must have, which is helpful for determining the set of all roots.

\subsection{The Bisection Method}
\noindent
The bisection method is one of the earliest methods used to find the roots of the equation $f(x)=0$ \cite{solanki2014role}. This method can only be applied to a function $f$ if it continuous on an interval $[a,b]$, and $f(a)$ and $f(b)$ have opposite signs. In this case, $a$ and $b$ are said to form a \textit{bracket} for the function. On $[a,b]$, then, the function $f$ has at least one root by the intermediate value theorem. More precisely, a corollary of this theorem known as \textit{Bolzano's theorem} states that a continuous $f$ on $[a,b]$ with $f(a)$ and $f(b)$ of opposite signs must have at least one $c \in (a,b)$ such that $f(c)=0$. For this reason, the bisection method is guaranteed to find a root, provided the continuity and sign requirements are met and the process is not limited by a maximum number of iterations or steps.

Now we explain the bisection algorithm. First, we check that the function values at the endpoints have opposite signs. If they have the same sign (that is, $f(a)\cdot f(b) >0$), then the method cannot be applied. Note that if $f(a)$ or $f(b)$ are zero, then a root is found. Otherwise, we calculate the midpoint $m$ of the interval by \[m=\frac{a+b}{2}\] and check if it is a root. If it is not, then we check if $f(m)$ has the same sign as $f(a)$, in which case $a$ is replaced by $m$. Else, $b$ is replaced by $m$. A new midpoint is then calculated and the process repeats.

This algorithm is written in Python in Appendix \ref{manualalgorithms}. To our function called \texttt{bisection} we give the bracket as well as two other variables: tolerance and the maximum number of iterations. If the value of the function is extremely close to zero, then the algorithm should terminate since a root is found. Alternatively, if the width of the bracket becomes extremely small, the process should stop as well, as this is a usual test that indicates a root has been located \cite{naseem2022novel}. Thus, the lines
\begin{verbatim}
    if abs(f(mid)) < tolerance or abs(b-a) < tolerance:
      return mid, points
\end{verbatim}
show how the function stops and returns the approximate root \texttt{mid} (as well as a list of previous approximations called \texttt{points}).

There are other methods that use a bracket and the intermediate value theorem to ensure a root will be found if it is known to be in an interval. Brent's method is a hybrid method that switches between the bisection method and two other methods to (ideally) locate the root faster. TOMS748, an algorithm published in ACM Transactions on Mathematical Software (TOMS), is another modern method that uses multiple other algorithms for faster convergence, typically \cite{alefeld1995algorithm}.

\subsection{Newton's Method}
\noindent
Another popular root-finding method was developed by Newton and first printed in 1685, before Joseph Raphson reformulated the technique in 1690 \cite{sutherland1989finding}. Newton's method (or the Newton-Raphson method) starts with an initial guess $x_0$ and forms a better approximation $x_1$ of the root for a real-valued function $f$ by drawing the tangent line of the function at $(x_0, f(x_0))$ and letting $x_1$ be the point where the tangent line meets the $x$-axis \cite{sutherland1989finding}. We repeat this process iteratively, using the rule \[x_{i+1} = x_i - \frac{f(x_i)}{f'(x_i)}.\]

This key rule is easily written in Python as
\begin{verbatim}
    xnew = x0 - (f(x0)/df(x0))
\end{verbatim}
and we loop over the maximum number of iterations until a root is found within the given tolerance. While this method is known to usually converge quick, there are three main difficulties we can identify at this stage. First, and unlike the bisection method, convergence upon a root or a desired root is not guaranteed. This is an \textit{open method}, and we see some of its associated struggles in the Results section. Second, the rule requires that the function has a derivative $f'$. Third, the derivative must not be zero at the initial guess or subsequent approximations, or else division by zero would occur. Our algorithm acknowledges this final difficulty and will not return a root if a derivative of zero occurs:
\begin{verbatim}
    if abs(df(x0)) < 1e-10:
      print("Error: Derivative is zero.")
      return None, points
\end{verbatim}

Secant method is another open method that uses a secant line instead of a tangent line. Its rule \[x_{i+1} = x_i - f(x_i)\frac{x_i - x_{i-1}}{f(x_i) - f(x_{i-1})}\] requires two points instead of a derivative. Our program for comparing all of the methods uses the \texttt{SciPy} package. Each of the root-finders returns the approximate root (if found) followed by a \texttt{RootResults} object from which we can extract important data and use to plot graphs. Omitting the derivative in the Newton's method function causes it to run secant method instead:
\begin{verbatim}
    root, secantinfo = scipy.newton(f, x0, maxiter=500, full_output=True)
\end{verbatim}
Likewise, Halley's method is a variation that uses the rule \[x_{i+1} = x_i - \frac{2f(x_i)f'(x_i)}{2[f'(x_i)]^2 - f(x_n)f''(x_n)}\] and must be supplied a first and second derivative:
\begin{verbatim}
    root, halleyinfo = scipy.newton(f, x0, fprime=df, fprime2=ddf, 
                                    maxiter=500, full_output=True)
\end{verbatim}

The bisection method and Newton's method have been programmed manually in order to more precisely examine how they work. In a second Python notebook, all six of the methods mentioned have been implemented using \texttt{SciPy}. This sets a great framework for studying their strengths, efficiency, and how they sometimes fail.

\section{Results}

\subsection{Visualizing Root-Finding Algorithms}
\noindent
First, we want to visualize our two main methods in action. We do this by plotting the root approximations after every iteration. Consider the polynomial $p_1(x) = x^3-7x$. In this opening example, the three roots, $0, \sqrt{7}$, and $-\sqrt{7}$, are easily found algebraically.

\begin{figure}[h]
    \centering
    \includegraphics[width=0.75\linewidth]{images/greatstart.png}
    \caption{Typical root-finder behaviour on the polynomial $p_1(x) = x^3-7x$ with $x_0=-4.5$ for Newton's method and bracket $[2, 4]$ for the bisection method.}
    \label{greatstart}
\end{figure}

With our tolerance set to $10^{-12}$, both root-finders perform well and Newton's method takes only 7 iterations. It is usual for Newton's method to converge quickly (i.e., requiring less iterations) when the initial guess is close. Bisection converges less quickly in this example, with guesses on either side of the (true) root before the algorithm reaches the stopping criteria. Note that in this example the bisection method and Newton's method produce accurate, usable decimal approximations of $\sqrt{7}$ and $-\sqrt{7}$, respectively.

Suppose we want to determine the set of all real roots of the polynomial \[p_2(x) = x^6 - 2x^5 - 3x^4 + 4x^3 + 5x^2 + x - 7\] with real coefficients. Numerical methods are highly useful here. By the Fundamental Theorem of Algebra, there are 6 complex roots, some of which may be real. With the polynomial written in descending order as above, we count 3 changes in the signs of the leading coefficients. Thus, by Descartes' Rule of Signs, there are 3 or 1 positive real roots (as well as 3 or 1 negative real roots). Testing some small positive values of $x$ we find that
\begin{align*}
    p_2(1) &= -1 <0,\\
    p_2(1.5) &= 0.265625 >0,\\
    p_2(2) &= -1 <0,\\
    p_2(2.5) &= 20.890625 >0.
\end{align*}

By Bolzano's theorem, then, there are roots on the intervals $(1, 1.5), (1.5, 2)$, and $(2, 2.5)$. The bisection method can easily find (any of) these, as shown in Figure \ref{newpolynomial}.

\begin{figure}[h]
    \centering
    \includegraphics[width=0.75\linewidth]{images/newpolynomial.png}
    \caption{The degree six polynomial $p_2(x)$ with bracket $[1, 1.5]$ and $x_0=-1.75$.}
    \label{newpolynomial}
\end{figure}

If we examine negative $x$-values, we find that there is only one sign change and that it is somewhere around $-1.5$ (Newton's method is then well-suited to approximate this root). We conclude that there are 3 positive real roots, 1 negative real root (of multiplicity 1), and that the remaining 2 roots are non-real. Our root-finders determine the approximate set of real roots to be \[ \{-1.484,\ 1.175,\ 1.582,\ 2.105\}. \] Clearly, a combination of theory and algorithm selection helps solve problems like this.

\subsection{Difficulties in Root-Finding}
\noindent
Text.

\subsection{Wilkinson's Polynomial}
\noindent
Text.

\section{Conclusion}
\noindent
Text.

\addcontentsline{toc}{section}{Acknowledgements}
\section*{Acknowledgements}
\noindent
Text.

\addcontentsline{toc}{section}{References}
% \begin{thebibliography}{40}

\bibitem{item1} Text.

\bibitem{item2} Text.

\end{thebibliography}

\newpage
\bibliographystyle{plain}
\bibliography{mybib}

\newpage
\appendix

\section{Appendix A}\label{manualalgorithms}
\noindent
Text.

\section{Appendix B}
\noindent
Text.

\end{document}