\documentclass[12pt]{article}
\usepackage{amsmath, amssymb, amsfonts, amsthm}
\usepackage{graphicx, tabularx, geometry, xcolor, minted}
\usepackage{caption, float}
\usepackage{setspace, lipsum}

\usepackage{tikz, pgfplots}
\pgfplotsset{compat=1.18}

\geometry{top=1 in, bottom=1 in, left=1 in, right=1 in}

% \onehalfspacing

\theoremstyle{definition}
\newtheorem{theorem}{Theorem}
\newtheorem{definition}{Definition}
\newtheorem{lemma}{Lemma}
\newtheorem{proposition}{Proposition}
\newtheorem{remark}{Remark}
\newtheorem{corollary}{Corollary}
\newtheorem{problem}{Problem}
\newtheorem{example}{Example}

% \setlength\parindent{0 pt}

\begin{document}

\begin{titlepage}
    \begin{center}
        \vspace*{2 cm}

        \begin{spacing}{2}
            {\Large \textbf{Polynomial Root-Finding Methods: \\ Theory, Weaknesses, and Wilkinson's Polynomial}}
        \end{spacing}

        \vspace*{1 cm}

        {\large Joel Penney} \par

        \vspace*{0.25 cm}
        
        \texttt{jscottp@mun.ca}

        \vfill

        \begin{abstract}
            This paper studies six algorithms used to solve root-finding problems involving single-variable polynomials. Specifically, we consider theoretical foundations such as the intermediate value theorem as well as the details of iterative rules to explain the differences between several bracketed and open methods. The six algorithms are also compared under challenging circumstances such as flat regions and Wilkinson's polynomial. We find that no algorithm performs consistently better than all of the others, although TOMS748 and Halley's method are often able to locate roots quickly. As we demonstrate, the structure of the root-finding problem influences the chosen approach, relevant theory considered, and the methods applied to efficiently find roots.
        \end{abstract}

        \vfill

        \begin{spacing}{1.5}
            Memorial University of Newfoundland \par
            Department of Mathematics and Statistics \par
            MATH 3030: Mathematical Inquiry II
        \end{spacing}

        \vspace*{1.25 cm}

        January 29, 2026

        \vspace*{0.25 cm}
    \end{center}
\end{titlepage}

% \tableofcontents
% \newpage

\section{Introduction}
\noindent
Text.

\section{Methods}

\subsection{Preliminaries}
\noindent
Suppose we know a function $f$ and want to determine the value or values of $x$ such that \[f(x)=0.\]

We may first try analytical methods to determine $x$, which is often convenient if $f$ is sufficiently basic, such as a polynomial of degree 1 or 2. Frequently, however, our function $f$ is more complicated or sophisticated, and analytical methods are impractical or unavailable. Instead, numerical methods are often imposed on \textit{root-finding problems} such as the one described. Root-finding algorithms (or root-finding methods) are used to approximate these roots or zeros, with the requirement that the function be continuous. Additionally, solving the equation \[g(x)=h(x)\] is equivalent to finding the roots of $f(x)=g(x)-h(x)$, which shows how widespread root-finding problems are and how important root-finding methods can be. 

\begin{definition}
    A \textit{polynomial of one variable}, denoted $p(x)$, is an expression that can be written in the form \[a_nx^x + a_{n-1}x^{x-1} + \cdots +a_1x + a_0\] where $a_n, ..., a_0$ are \textit{coefficients}, $a_n$ is nonzero, $x$ is a variable, and $n$ is the degree \cite{kalantari2008polynomial}. If the coefficients are real and the variable is restricted to the real numbers, then the expression is called a \textit{real polynomial}. The equation $p(x)=0$ is called a \textit{polynomial equation}.
\end{definition}

Each polynomial of one variable (or just polynomial for short) is continuous everywhere.

\begin{theorem} [Fundamental Theorem of Algebra]
    Every nonzero polynomial of one variable with complex coefficients has exactly $n$ complex roots, assuming multiplicity is counted.  
\end{theorem}

Many proofs of this famous theorem exist. They can be found in  \cite{steed2015proofs}, for example. Although we are primarily interested in \textit{real} roots, the theorem indicates exactly how many (complex) roots a polynomial must have, which is helpful for determining the set of all roots.

\subsection{The Bisection Method}
\noindent
Text.

\subsection{Newton's Method}
\noindent
Text.

\section{Results}
\noindent
Text.

\section{Conclusion}
\noindent
Text.

\addcontentsline{toc}{section}{Acknowledgements}
\section*{Acknowledgements}
\noindent
Text.

\addcontentsline{toc}{section}{References}
% \begin{thebibliography}{40}

\bibitem{item1} Text.

\bibitem{item2} Text.

\end{thebibliography}

\newpage
\bibliographystyle{plain}
\bibliography{mybib}

\newpage
\appendix

\section{Appendix A}
\noindent
Text.

\section{Appendix B}
\noindent
Text.

\end{document}