\documentclass[12pt]{article}
\usepackage{amsmath, amssymb, amsfonts, amsthm}
\usepackage{graphicx, tabularx, geometry, xcolor, minted}
\usepackage{caption, float}
\usepackage{setspace, lipsum}

\usepackage{tikz, pgfplots}
\pgfplotsset{compat=1.18}

\geometry{top=1 in, bottom=1 in, left=1 in, right=1 in}

% \onehalfspacing

\theoremstyle{definition}
\newtheorem{theorem}{Theorem}
\newtheorem{definition}{Definition}
\newtheorem{lemma}{Lemma}
\newtheorem{proposition}{Proposition}
\newtheorem{remark}{Remark}
\newtheorem{corollary}{Corollary}
\newtheorem{problem}{Problem}
\newtheorem{example}{Example}

% \setlength\parindent{0 pt}

\begin{document}

\begin{center}
        \begin{spacing}{2}
            {\Large \textbf{Project 1 Research Proposal}}
        \end{spacing}

        \vspace*{0.25 cm}

        {\large Joel Penney} \par

        \vspace*{0.25 cm}
        
        \texttt{jscottp@mun.ca}

        \vspace*{0.75 cm}

        January 13, 2026
    \end{center}

\subsection*{Project Topic}

In this project, I aim to study root-finding algorithms on polynomials of one variable. In particular, I want to examine the bisection method, Newton's method, and more advanced methods used frequently in numerical analysis today. This project will start by reviewing the theoretical foundations of root-finding. Then, I will implement the algorithms and determine strengths and weaknesses of each method.

\subsection*{Key Research Questions}

\begin{itemize}
    \item What is the underlying theory behind polynomial roots, root-finding, and methods such as the bisection method? How do the various root-finding algorithms work?
    \item What are the strengths and weaknesses of each method? What circumstances is each method best suited for? Where and why do they sometimes fail?
\end{itemize}

\subsection*{Planned Methodology}

After literature review and narrowing down exactly which methods to study, I will write code in \texttt{Python} that uses these methods to find the roots of a given polynomial. To compare each method, I will likely have to track the number of iterations, error, or other measures of success and efficiency. I plan to use Jupyter notebooks and perhaps the \texttt{SciPy} package to some degree.

\subsection*{Link to Version Control Repository}

Text.

\subsection*{Preliminary References}

Text.

\end{document}